\documentclass[tikz,convert=false]{standalone}

\usepackage{tikz}
\usepackage{verbatim}
\usepackage[active,tightpage]{preview}
\usetikzlibrary{decorations.pathreplacing}

\SetSymbolFont{letters}{bold}{OML}{cmbr}{bx}{it}
\renewcommand{\familydefault}{\sfdefault}


\usepackage{xparse}% http://ctan.org/pkg/xparse
\usepackage{etoolbox}% http://ctan.org/pkg/etoolbox
\newcounter{listtotal}\newcounter{listcntr}%
\NewDocumentCommand{\events}{o}{%
  \setcounter{listtotal}{0}\setcounter{listcntr}{-1}%
  \renewcommand*{\do}[1]{\stepcounter{listtotal}}%
  \expandafter\docsvlist\expandafter{\namesarray}%
  \IfNoValueTF{#1}
    {\namesarray}% \names
    {% \names[<index>]
     \renewcommand*{\do}[1]{\stepcounter{listcntr}\ifnum\value{listcntr}=#1\relax##1\fi}%
     \expandafter\docsvlist\expandafter{\namesarray}}%
}

\usetikzlibrary{positioning, decorations.markings}

\begin{document}

\begin{tikzpicture}
[invisible/.style={circle,draw=none,fill=none,inner sep=0pt,minimum size=0mm,line width=0mm}]

\newcommand{\namesarray}{Hebb Rule,Perceptron,Adaline,Minsky and Papert criticism,Paul Werbos found backpropagation,Neocognitron,Backpropagation,LeNet5,First \textit{``Deep''} learning algorithm,ReLU (AlexNet),MaxOut}



\def\years{{1949,1957,1960,1969,1974,1980,1986,1998,2006,2012,2013}}
\def \startYear{1949}
\def \endYear{2016}

\def \factor{0.32}
\def \horizontalOffset{0}
\def \verticalOffset{-0.5}


% Drawing Main Line
\pgfmathsetmacro\xEnd{(\endYear-\startYear)*\factor+\horizontalOffset}
\draw ({\horizontalOffset},{0}) -- ({\xEnd},{0});

\node[invisible] at ({\horizontalOffset},-1.25) {};

\def \numYears {10}
% Drawing years
\foreach \i in {0,...,\numYears} {
	\pgfmathparse{\years[\i]} \let\year\pgfmathresult			
        \pgfmathsetmacro\x{(\year-\startYear)*\factor+\horizontalOffset}
        \pgfmathsetmacro\y{\verticalOffset}
        \node[rotate=90,invisible] at ({\x},{\y-0.2}) {\textbf{\year}};
	\node[invisible] (\i) at ({\x},{\y+0.85}) {};
	\node [rotate=45,right=0cm of \i,draw=white,fill=none,inner sep=0pt,minimum size=0mm,line width=0mm] {\events[\i]};
        \draw (\x,0.15) -- (\x,-0.15);
}

\end{tikzpicture}


\end{document}
